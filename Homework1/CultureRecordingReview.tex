\documentclass[11pt]{article}
\usepackage{fullpage}
\usepackage{times}
\usepackage{amssymb}
\begin{document}
\begin{center}
{\large {\bf CIS 6930/4930 Homework 1: Culture Recording Review}}\\
{\normalsize {\bf Review of: "Brain-Derived Neurotrophic Factor Mediates the ActivityDependent
Regulation of Inhibition in Neocortical Cultures"}}\\
Amanda Baker \\
February 2nd, 2016 \\
\end{center}

\paragraph{Results}
When the cultures were treated with TTX (to represent being deprived of visual stimulation)
the number of GABA-positive neurons decrease relative to the control.  However,
when the cultures were treated with brain-derived neurotrophic factor (BDNF) and
the same quantity of TTX, significantly
fewer GABA-positive neurons died over the course of treatment.  Additionally, K252a,
an endogenous neurotrophin, was added to a solution used to incubate the cultures
for 2 days.  This resulted in a similar decrease in the percentage of GABA-positive
neurons.  While the cells were being treated with TTX, they would not produce spikes.
However, after washing for 30 minutes, the cells would return to spiking and had
an increased firing rate.  \\

\paragraph{Advances}
The paper concludes that because the activity-dependent regulation of circuit excitability
is mediated through the regulation of BDNF, that brain-derived neurotrophic factor
is in control of cortical excitability. \\

\paragraph{Experimental Design}
The cultures for this experiment were prepared from the visual cortices of 4 to 6
week old rat pups.  The rats were first anesthetized and decapitated, then the brains
were removed and a piece of the visual cortex removed and minced.  The tissue was
processed through multiple chemical solutions before being plated on culture dish
and incubated for 1 to 2 weeks.  The current-clamp recordings measured the potential
across the membrane while injecting current, and the voltage-clamp recordings measured
current while holding the potential difference between the inside and outside of
the cell.  Using these two techniques together allows the author to compare the
results of each to determine whether controlling either current or voltage unexpectedly
impacts the results of the experiment.  Additionally, the experimenters process
the cultures for indirect immunofluorescence against GABA by bathing the cultures
again in multiple solutions, ultimately making the GABA-positive neurons fluoresce. \\

\paragraph{Weaknesses}

\paragraph{Interpretation}

\paragraph{Further Reading} \tiny{.}\\
\normalsize
Dark-rearing/TTX injection leads to decrease of GABAergic neurons in V1: \\
\hspace*{0.5in}Hendry SHC, Jones EG (1986) Reduction in number of immunostained GABAergic neurones
in \\
\hspace*{0.5in}deprived-eye dominance columns of monkey area 17. Nature 320:750–753. \\
BDNF expression with light: \\
\hspace*{0.5in} Isackson PJ, Huntsman MM, Murray KD, Gall CM (1991) BDNF mRNA expression is increased \\
\hspace*{0.5in} in adult rat forebrain after limbic seizures: temporal patterns of induction distinct from NGF.\\
\hspace*{0.5in} Neuron 6:937–948.\\

\end{document}
