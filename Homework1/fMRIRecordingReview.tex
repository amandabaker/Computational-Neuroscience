\documentclass[11pt]{article}
\usepackage{fullpage}
\usepackage{times}
\usepackage{amssymb}
\begin{document}
\begin{center}
{\large {\bf CIS 6930/4930 Homework 1: fMRI Recording Review}}\\
{\normalsize {\bf Review of: "The Fusiform Face Area: A Module in Human Extrastriate Cortex
Specialized for Face Perception"}}\\
Amanda Baker \\
February 2nd, 2016 \\
\end{center}

\paragraph{Results}
Of the 15 subjects who were ultimately included in the analysis of part I, 12 displayed
distinct activation of fusiform region, and 3 did not have any area with distinctly
stronger MR signal when exposed to faces than at rest.  Part II suggests that the
area determined in part I activates better to two toned faces than two toned nonfaces.
Part III similarly found that this area responds better to two faces tests than
the nonface tests.

\paragraph{Advances}
The fact that across subjects part of the fusiform area activates in response to
faces, signifies that the face is a significant feature for recognition.

\paragraph{Experimental Design}
20 participants were included in the experiment which was broken into 3 parts. In
part I, subjects were shown images of faces during fMRI in order to determine a
centralized location in the occipitotemporal area. These images were roughly 300x300
pixels and displayed for 670ms at a time, presenting 7 sets of 45 photographs during
the 5 minute 20 scan.  In part II, a set 5 participants that displayed a distinct fusiform
face area in part I were subjected to two types of images: one of intact two toned
faces, and one of scrambled two tone faces (so that the image no longer resembles
a face).  This allows the experimenter to compare stimuli to determine whether the
face itself is causing the reaction of the fusiform face area, or if there is another
cause.  In part III, another group of 5 participants who had a clear fusiform face
area were exposed to another set of photos.  This set compared a partially occluded
face to a human hand, allowing the experimenters to determine whether it was a face
or just a human feature causing the exication. This part was conducted in two ways,
first just by displaying the images, then by presenting during a 1-back task
in which subjects searched for consecutive repitions of identical stimuli.

\paragraph{Weaknesses}
Only 20 test subjects were used, 5 of which were not included in analysis due to
head movement or other artifacts.  Furthermore, parts II and III only included
5 of the subjects who patricipated in part I, which is not a significant sample
size.

\paragraph{Interpretation}


\paragraph{Further Reading} \tiny{.}\\
\normalsize
More research on facial recognition: \\
\hspace*{0.5in} Behrmann M, Winocur G, Moscovitch M (1992) Dissociation between mental imagery and object \\
\hspace*{0.5in} recognition in a brain-damaged patient. Nature 359:636–637.\\


\end{document}
