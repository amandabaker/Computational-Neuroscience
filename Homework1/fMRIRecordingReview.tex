\documentclass[11pt]{article}
\usepackage{fullpage}
\usepackage{times}
\usepackage{amssymb}
\begin{document}
\begin{center}
{\large {\bf CIS 6930/4930 Homework 1: fMRI Recording Review}}\\
{\normalsize {\bf Review of: "The Fusiform Face Area: A Module in Human Extrastriate Cortex
Specialized for Face Perception"}}
\end{center}

\textbf{Results} \\

\textbf{Advance} \\

\textbf{Experimental Design} \\
The cultures for this experiment were prepared from the visual cortices of 4 to 6
week old rat pups.  The rats were first anesthetized and decapitated, then the brains
were removed and a piece of the visual cortex removed and minced.  The tissue was
processed through multiple chemical solutions before being plated on culture dish
and incubated for 1 to 2 weeks.  The current-clamp recordings measured the potential 
across the membrane while injecting current, and the voltage-clamp recordings measured
current while holding the potential difference between the inside and outside of
the cell.  Using these two techniques together allows the author to compare the
results of each to determine whether controlling either current or voltage unexpectedly
impacts the results of the experiment.  Additionally, the experimenters process
the cultures for indirect immunofluorescence against GABA by bathing the cultures
again in multiple solutions, ultimately making the GABA-positive neurons fluoresce. \\

\textbf{Weaknesses} \\

\textbf{Interpretation} \\

\textbf{Further Reading} \\
Dark-rearing/TTX injection leads to decrease of GABAergic neurons in V1: (Hendry
and Jones, 1986) \\
BDNF expression with light (Isackson et al., 1991;) \\


\end{document}
