\documentclass[11pt]{article}
\usepackage{fullpage}
\usepackage{times}
\usepackage{amssymb}
\begin{document}
\begin{center}
{\large {\bf CIS 6930/4930 Homework 1: In-Vivo Recording Review}}\\
{\normalsize {\bf Review of: "Functional Micro-Organization of Primary Visual Cortex: Receptive
Field Analysis of Nearby Neurons"}}
\end{center}

The results state that a pair of neighboring simple neurons in the striate cortex
can have similar receptive fields (RF) and responses to stimuli.  This correlation is
strongest across time (T) and the direction parallel to preferred orientation of
the neuron (Y), but not across the direction perpendicular to preferred orientation
of the neuron (X).  Due to the uniformity across X, there is also little correlation
between the pair of neurons accross the X-T and X-Y-T dimensions.\\

The paper concludes that neurons may have overlapping but dissimilar RFs to reduce
noise via pooling. It also concludes that response variables such as RF size, response
latency and duration, and temporal frequency are clustered, implying that the striate
cortex is more complex than the assumed models.  Additionally, the shape of RFs are
diverse and have no apparent clustering.




\end{document}
