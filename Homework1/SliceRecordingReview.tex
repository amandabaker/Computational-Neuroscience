\documentclass[11pt]{article}
\usepackage{fullpage}
\usepackage{times}
\usepackage{amssymb}
\begin{document}
\begin{center}
{\large {\bf CIS 6930/4930 Homework 1: Slice Recording Review}}\\
{\normalsize {\bf Review of: "The NMDA-to-AMPA Ratio at Synapses Onto Layer 2/3 Pyramidal Neurons
Is Conserved Across Prefrontal and Visual Cortices"}}\\
Amanda Baker \\
February 2nd, 2016 \\
\end{center}

\paragraph{Results}
According to the paper, miniature excitatory post-synaptic currents (mEPSCs) had
a similar average amplitude, and the NMDA/AMPA ratio also did not differ significantly.
Similarly, the evoked EPSC tests found matching results despite the difference in
technique.

\paragraph{Advances}
This data combats claims that areas specialized for persistant activity should have
a greater contribution from NMDARs.  Rather, the same ratio seems to spread across
the cortex.

\paragraph{Experimental Design}
The authors measured the NMDA/AMPA ratio in brain slices of 2-4 week old rats, comparing
the cells of the prefrontal cortex (PFC) to those of the visual cortex (V1).  Signals
were sampled at 5-10 kHz.  The authors chose to measure both mEPSCs without electrically
stimulating the tissue as well evoked EPSCs with extracellular stimulation.  The mEPSC
recording offered results which had been less altered than the alternative, but at
the sacrifice of noise.  Evoked EPSCs offer a better signal-to-noise ratio, but add
another variable to balance.  Choosing to use both, the author has the opportunity
to compare results, which in this case, were similar between the two experimental
procedures.

\paragraph{Weaknesses}
The author admits the experiment's limitations by stating that similar ratio of NMDA
to AMPA may be regional within each cortex and that these regional differences may
be lost in averaging.  Additionally, the experiment only evaluates mEPSCs with an
AMPA component, possibly leaving out NMDA-only synapses.

\paragraph{Interpretation}

\paragraph{Further Reading}
Coregulation of AMPA and NMDA of excitatory postsynaptic currents: \\
\hspace*{0.5in} Watt AJ, van Rossum MC, MacLeod KM, Nelson SB, and Turrigiano GG. Activity coregulates \\
\hspace*{0.5in} quantal AMPA and NMDA currents at neocortical synapses. Neuron 26: 659–670, 2000. \\
Contribution of NMDARs to excitatory synaptic transmission  \\
\hspace*{0.5in} Burgard EC and Hablitz JJ. Developmental changes in NMDA and nonNMDA receptor-mediated \\
\hspace*{0.5in} synaptic potentials in rat neocortex. J Neurophysiol 69: 230–240, 1993. \\

\end{document}
